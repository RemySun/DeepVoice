\documentclass[11pt,english]{beamer}

\usepackage[T1]{fontenc}
\usepackage[utf8]{inputenc}
\usepackage[main=french,english]{babel}
\usepackage{hyperref}
\usepackage{tikz}
\usepackage{ulem}
\usepackage{xcolor, colortbl}
\usepackage{mathtools}

\usetheme{AnnArbor}
\usecolortheme{beaver}

\title{DeepVoice}
\subtitle{Extracting meaningful signal representation for Speaker Recognition
  using deep architectures}

\author{Rémi~Hutin, Rémy Sun, Raphaël Truffet\\ \\Encadrants : Guillaume Gravier et Vedran Vukoki\'c}


\institute{ENS Rennes, IRISA}

\date{8 décembre 2016}


\AtBeginSection[]{
    \begin{frame}
        \frametitle{Sommaire}
        \tableofcontents[currentsection,hideothersubsections]
    \end{frame} 
}


\begin{document}

\begin{frame}
    \titlepage
\end{frame}

\begin{frame}
    \frametitle{Sommaire}
    \tableofcontents[hideallsubsections]
\end{frame}


\section{Signal representation for speaker recognition}

\begin{frame}
  \frametitle{Cesptral alanysis}
  
\end{frame}

\begin{frame}
  \frametitle{Cepstral space}
  
\end{frame}

\begin{frame}
  \frametitle{Probabilistic modeling}
  
\end{frame}

\begin{frame}
  \frametitle{supervectors}
  
\end{frame}

\begin{frame}
  \frametitle{i-vectors}
  
\end{frame}

\begin{frame}
  \frametitle{i-vectors}
  
\end{frame}

\begin{frame}
  \frametitle{i-vectors}
  
\end{frame}

\begin{frame}
  \frametitle{transition}
  
\end{frame}

\begin{frame}
  \frametitle{transition}
  
\end{frame}

\section{Deep learning}

\begin{frame}
  \frametitle{Use of DNNs}
  
\end{frame}

\begin{frame}
  \frametitle{Formal neuron}
  
\end{frame}

\begin{frame}
  \frametitle{Neural network}
  
\end{frame}

\begin{frame}
  \frametitle{Autoencoder}
  
\end{frame}

\begin{frame}
  \frametitle{Autoencoder}
  
\end{frame}

\begin{frame}
  \frametitle{i-vector 2.0}
  
\end{frame}

\section{Methodology}

\begin{frame}
  \frametitle{Supplanting i-vectors}
  
  \begin{columns}
  \column{0.44\textwidth}  
  
\only<1>{
\begin{equation*}
    \begin{bmatrix}
      G_0^{0,0} \\   G_0^{0,1} \\ ... \\ G_{0}^{0,255} \\ G_0^{1,0} \\ .. \\ G_{0}^{N,255}
    \end{bmatrix}
\begin{bmatrix}
      G_1^{0,0} \\   G_1^{0,1} \\ ... \\ G_{1}^{0,255} \\ G_1^{1,0} \\ .. \\ G_{1}^{N,255}
    \end{bmatrix}
...
\begin{bmatrix}
      G_M^{0,0} \\   G_M^{0,1} \\ ... \\ G_{M}^{0,255} \\ G_M^{1,0} \\ .. \\ G_{M}^{N,255}
    \end{bmatrix}
  \end{equation*}
M supervectors
}

\only<2>{
  \begin{equation*}
    \begin{bmatrix}
      G_0^{0,0} \\   G_0^{0,1} \\ ... \\ G_{0}^{0,255} \\ G_0^{1,0} \\ .. \\ G_{0}^{N,255}
    \end{bmatrix}
  \end{equation*}
  Supervector from signal spoken by \textbf{someone}
}
  \column{0.5\textwidth}  
  What is needed to extract i-vectors ?
  \begin{itemize}
  \item Supervectors
  \item Training data (Labels on supervectors)
  \end{itemize}
  We will use the \textbf{exact same data}
  \end{columns}

\end{frame}

\begin{frame}
  \frametitle{Processed data}

  \begin{columns}
    \column{0.4\textwidth}
    \begin{itemize}
    \item Raw data: 15311 numeric soud files from BFMTV with labeled speakers
    \item Input: 3 678 470 supervectors

    \end{itemize}

    \column{0.4\textwidth}
  \end{columns}
  
\end{frame}

\begin{frame}
  \frametitle{Intermediate vector evaluation}
  
\end{frame}

\begin{frame}
  \frametitle{Intermediate vector evaluation}
  
\end{frame}

\section{Discussion}

\begin{frame}
  \frametitle{Goal}
  
\end{frame}

\begin{frame}
  \frametitle{Expected issues}
  
\end{frame}

\end{document}



