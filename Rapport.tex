\documentclass[conference]{IEEEtran}
\usepackage[T1]{fontenc}
\usepackage[utf8]{inputenc}

%\usepackage{fullpage}           %
\usepackage[french, english]{babel}
\usepackage{graphicx}
\usepackage{todonotes}

\usepackage[backend=biber]{biblatex}
\bibliography{Biblio}{}

\title{PROJ: DeepVoice}

\author{Remi Hutin, Remy Sun, Raphael Truffet \\
  \{Remi.Hutin, Remy.Sun, Raphael.Truffet\}@ens-rennes.fr \\
  Département informatique, ENS Rennes \\
  Campus de Ker lann, Bruz, France
\and
  Guillaume Gravier \\
  guig@irisa.fr\\
  Linkmedia project, INRIA \\
  Campus de Beaulieu, Rennes, France
 }
\begin{document}

\maketitle

\begin{abstract}
  To carry out a speaker recognition task, ie to identify the locutor in an
audio signal, one must first obtain a serviceable representation
of said signal. Improvements upon the raw numeric signal have been made over the years, one
of which is the creation of a Gaussian Mixture Model supervector modeling the
probabilistic distribution of the signal's
spectral features. Due to the enormous size of such supervectors, condensed
forms such as i-vectors have been created to provide usable data for application
tasks. Deep learning techniques have seen significant success in many
classification tasks due to their ability to
build upon successive layers of data representation. We will
strive to explore the possibility of using such techniques to acquire a
representation of supervectors that is at least competitive with i-vectors.
\end{abstract}

\section{Introduction}

Speaker recognition refers to the all too important identification of a locutor
from an unlabeled audio signal. However, the resolution of this essential issue
is no trivial matter as it raises a number of questions along the way. The most
immediate one, is the manner in which the data will be represented: raw numeric
signal, fourrier transform, spectral representation? 

Years of reasearch into the issue have yielded a solution called
\texttt{supervectors} which provide a probabilistic representation of the
numeric signal's spectral features. Those supervectors however present the
significant downside of being enormous and therefore evolving in a very sparse
representation space, which makes them unfit for applications. What recent works
have had success with is the extraction of more condensed representations from
those supervectors named \texttt{i-vectors}.

Deep Learning techniques have had tremendous success in a number of fields
ranginf from computer vision (\cite{lecun1998gradient}) to natural language
processing (\cite{bordes2012joint}) by building upon successive layers of data
representation and inferring hierachical dependencies within the data. It seems
to us that such architectures would be especially adapted to the extraction of
intermediate representations from \texttt{supervectors} and might provide
representations that outperform the current state of the art \texttt{i-vectors}.

\section{Sound signals}

\subsection{Cepstral analysis}

The speech signal of a locutor is hardly suitable for statistical modeling or the calculation of a distance.
In order to obtain a representation which is more compact and less redundant, we use a cepstral representation of the speech.

The cepstrum of a signal $x(t)$ is defined by :

$$C(x(t)) = \mathcal{F}^{-1}(\ln(|\mathcal{F}(x(t))|)$$

where $\mathcal{F}$ is the Fourier transform.


We can analyze a signal locally by applying a window, whose duration is shorter than the signal. Then, we exctract a vector of cepstral coefficents of this part of signal. We repeat it for several windows, until the end of the signal is reached.

In the cepstral domain, the distance between two speech signals may be easily computed, using the Euclidean distance.



\subsection{Gaussian Mixture Model (GMM) and supervectors}

A Gaussian Mixture Model (GMM) is a probabilistic model used to approximate a distribution of random variables as a sum of normal distributions. 
Here, we suppose that cepstral vectors of a signal follow a probality distribution that is specific to the given signal. This distribution is the one that we try to approximate with the GMM using a reference model. The supervector is the vector that gather means of all the normal distributions of the GMM.


\todo[inline]{Insert a figure}

\subsection{I-vectors}

\subsection{Previous works}

\section{Use of neuronal networks}

The recent success of deep learning techniques in various fields such as
computer vision (\cite{lecun1998gradient}) and natural language processing (\cite{bordes2012joint}) has sparked
many explorations of their usefulness in classification tasks on
\texttt{i-vectors} such as speaker recognition. Many architectures such as Deep
Belief Networks(\cite{DBLP:journals/corr/GhahabiH15},\cite{ghahabi2014deep}),
Deep Neural Networks
(\cite{DBLP:journals/corr/GhahabiH15},\cite{ghahabi2014deep}), Recurrent
Networks (\cite{DBLP:journals/corr/SaonSRK16}) or even a mix of
Deep neural networks and Support Vector machines (\cite{richardson2015deep}) have been tested and yielded
better results than previous techniques, though - to the best of our knowledge -
none directly tackled the issue of supervectors' intermediate representation.
Instead, all of the aforementionned work relied on pre-existing i-vector
extraction techniques. We will explore the possibility of extracting a
meaningful intermediate representation of supervectors through the use of deep architectures.

\subsection{Formal neuron}

\paragraph{Neuron?}
A neuron can be thought of as a function which takes an $n$-dimensional vector
$A$ as input and returns a scalar $e$ as output. This function typically has two
internal parameters which are a bias $b$ and a weight-matrix $W$. The function
starts by calculating $WA+b$ before using a non-linear activation function (such as sigmoid or tanh): $e=f(WA+b)$.

\paragraph{Adjusting the function}

Our endgoal is to have the neuron, and by extension the neural network, perform
a certain task. The formal neuron \og learns\fg by adjusting its function to perform better on
this designated task. For simplicity's sake, we will first explain how this
process - called \og backpropagation\fg{} - works with a single neuron.

For instance, suppose we have a bi-dimensional vector given as input and that we
want our neuron to return 1 if its two coordinates are the identical and -1 if
it is not. A natural way to evaluate how accurate our neuron is by looking at the
distance between its output $e$ and the desired result r : d(e)=|e-r|.

We want to modify our neuron/function to minimize this distance. That means
changing $e$, typically by gradient descent on function $d$ derivative. Here,
$\frac{\partial d}{\partial e} = r-e$, which means we want to \og move\fg{} $e$ in this
direction. To this end, we modify our function's two internal parameters $W$ and
$b$. $e$, and by extension $d$, can actually be seen as a function of those two
parameters: $d(W,b)=|e(W,b)-r|$. Therefore $\frac{\partial d}{\partial W}
   = \frac{\partial d}{\partial e}\frac{\partial e}{\partial W}$,
$\frac{\partial d}{\partial b}
   = \frac{\partial d}{\partial e}\frac{\partial e}{\partial b}$. We then only
   need to compute new internal parameters $W'$ and $b'$ with 

\begin{equation}
     W'=W + s\frac{\partial d}{\partial W}=W-s\frac{\partial d}{\partial e}\frac{\partial e}{\partial W}
\end{equation}

\begin{equation}
     b'=b + s\frac{\partial d}{\partial b}=b-s\frac{\partial d}{\partial e}\frac{\partial e}{\partial b}
\end{equation}

What we just demonstrated was a simple backpropagation algorithm called gradient
descent. This method is deeply flawed, but most state of the art backpropagation
methods find their origins in this humble algorithm.

\paragraph{Neuronal network?}

Typically, a neural \textbf{network} is made up of more than a single neuron. A
neural layer refers to multiple neurons working on the same input (or parts of
the same input) and producing an output that can be construed as some form of
concatenation of their respective outputs. This output can be in turn regarded
as an alternate representation of the input. Backpropagation for each neuron
works the same way as it would if it were the only neuron calculating.

The notion of \textbf{deep} learning comes from the fact that the alternate
reprensentation computed by one layer $A$ can be fed as input to another layer $B$.
This allows networks to infer multiple levels of representation, same as one
first processes simple geometrical forms before recognizing more complex
compositions. Backpropagation is straight-forwardly computed on layer $B$. It is
computed on layer $A$ by looking at $\frac{\partial d}{\partial input_B}$
instead of $\frac{\partial d}{\partial e}$
\subsection{Autoencoders}

An autoencoder (\cite{Hinton504}
) is a neuronal network with a particular architecture which is defined below.

\paragraph{A default task for a different goal}

Neural networks' ability to extract internal meaningful representation from the
original raw data is very enticing. Indeed, providing a meaningful
representation of the initial data is one of the biggest hurdles of classical
machine learning. However, neural networks need to be trained on a task, and
require labeled data to that end. 

Since we are no longer interested in the specific task needed to train the
network, it is possible to default to a task that is not interesting in itself
but might require the network to learn salient features of the data in order to
be completed. A very natural one is to match the output to the input, which
would not require man made labeling. Hopefully, the learned representation will
capture features specific to the input so that it may be reconstructed.

\paragraph{Structure}

Typically, the \textbf{input} is \textit{encoded} into a \textbf{latent
  representation} which is then \textit{decoded} into an \textbf{output} that
should match the input as closely as possible, hence the autoencoder
denomination. The main danger of such an approach is having the networks that
does not transform the input at all and yields a latent representation that is
no different to the raw input data. 

% A neuronal network that tries to reconstruct the input may learn the identity.
% But an autoencoder has a structure the makes it impossible. In fact, an
% autoencoder has at least one hidden layer that is smaller than the input. So an
% autoencoder can be decomposed as an encoder that transform the input to a
% smaller represatation, called latent representation or code, and a decoder that
% reconstruct the input from the latent representation.

\begin{figure}[!h]
    \centering
    \caption{Structure of an autoencoder}
    \includegraphics[width=7cm]{Autoencoder_structure.png}
    \label{autoencoder_structure}
\end{figure}


\paragraph{Why use autoencoders?}

Autoencoders can be used for denoising, using corrupted data as input and the original data as objective.

Autoencoders may also be very useful to learn a representation. In fact, the latent represention contains enough information to reconstruct the data.

In our context, autoencoders could be used to find a representation of a speaker, by giving to the autoencoder a supervector from a speaker as input, and an another supervector from the same speaker as objective. This idea is based on considerating the within-speaker variability as a noise, and using a denoising autoencoder. Repeating this with several pairs of supervectors, we may learn a representation of the speaker. 

Since the problem is symmetrical, we can perform the learning in both directions. One way to perform this training is to use an autoencoder that reconstruct each supervector from the other using tied weights.

\subsection{Tied weight autoencoder}

\paragraph{Architecture}

We talk about tied weight autoencoder when the same weights appears in two different places in the autoencoder. For example, in the architecture proposed in \cite{vukotic:hal-01314302}
(Fig. \ref{archi_vedran}), the weights in light blue in the upper part of the autoencoder are the same as the weights in light blue in the lower part. So, if the function that transform the upper first hidden layer to the upper part of the representation is $h_1(x) = f(W \times x + b_1)$, then  the function that transform the lower part of the representation to the next layer is $h_2(x) = f(W^T \times x + b_2)$. In this architecture, the pink weights are also tied.

\begin{figure}[!h]
    \centering
    \caption{Symmetrical and bidirectional learning architecture using tied weights}
    \includegraphics[width=5cm]{archi-vedran.pdf}
    \label{archi_vedran}
\end{figure}

\paragraph{Good results}

\section{Method}

As explained earlier, supervectors computed from Gaussian Mixture Models provide
useful features of a given signal. i-vectors have built upon this representation
to extract a more compact reprensentation that better reprensent the
specificities of the signal.

One task that naturally comes to mind is deciding whether two signals have been
said by the same person. Unfortunately, computing euclidian distance between the signals'
respective supervector or i-vector does not yield satisfactory results.
Therefore, we will use neural networks to extract an intermediate representation
of each signal that might be better suited to this particular problem.

\subsection{Prospective methodology}

\paragraph{Evaluating speaker-dependant similarities}

\todo[inline]{Talk about evaluation with guig. SVM? Neural classifier?}

\paragraph{Phasing out non-speaker dependant noise}

In order to extract a representation of supervectors that makes speaker
dependent features of the given signal more salient, we use a model derived from
the design of a denoising autoencoder. Indeed, we have no need for any
information pertaining to what is actually being said or what kind of microphone
was used: this is all noise to us. Therefore, if two supervectors originate from
the same speaker, it seems reasonnable to think of them as the same original \og
speaker\fg{} sound vector which has polluted by non-speaker dependant noise.

The latent representation thusly extracted should present salient features
specific to the studied speaker. An autoencoder trained that way could allow for
a sort of projection that we wish to study.

\paragraph{Prospective general model}

We will therefore train an autoencoder on supervectors. Given good results of
[Vedran's paper], we will tie all but the extreme weights to reflect the
symmetrical nature of the task at hand.

\subsection{Preliminary results}

In light of the difficulties inherent to the training of large neural networks,
a preliminary study on a small dataset has been conducted.

\subsubsection{Dataset}

\paragraph{Processing raw data}

15311 audio signals were extracted from BFMTV's various programs and labeled
with the name of their respective speaker. Those soundfiles were then processed
into 15311 supervectors of length 9311 (?) with the \texttt{AudioSeg} tool.
We then create 1800000 pairs of supervectors that originate from the same
speaker.

\paragraph{Input}

For each computed pair, we feed both of its suervectors to the neural network as
input. Therefore, we train the network on 3 600 000 supervectors of size 9311.

\paragraph{Output and task}

The neural network outputs a vector of size 9311 that we try to match as closely
as possible to the supervector the original input was paired with.

\appendix

\printbibliography




\end{document}
